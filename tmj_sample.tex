\documentclass{book}
\usepackage{tmj}
\usepackage[all,ps,arc]{xy}
\usepackage{amsmath,amssymb,amsfonts}
\usepackage{graphicx}

\newcommand{\A}{\mathbb{A}}
\newcommand{\AAA}{\mathcal{A}}
\newcommand{\B}{\mathbb{B}}
\newcommand{\BB}{\mathcal{B}}
\newcommand{\CC}{\mathbb{C}}
\newcommand{\C}{\mathcal{C}}
\newcommand{\D}{\mathcal{D}}
\newcommand{\E}{\mathcal{E}}
\newcommand{\PP}{\mathbb{P}}
\newcommand{\G}{\mathbb{G}}
\newcommand{\K}{\mathbb{K}}
\newcommand{\T}{\mathbb{T}}
\newcommand{\TT}{\mathcal{T}}
\newcommand{\N}{\mathbb{N}}
\newcommand{\X}{\mathbb{X}}
\newcommand{\Z}{\mathbb{Z}}
\newcommand{\cd}{\xymatrix}
\newcommand{\fracinline}[2]{\raisebox{0.4ex}{$#1$} / \raisebox{-0.7ex}{$#2$}}
\newcommand{\pb}[1][dr]{\save*!/#1-1.5pc/#1:(-1,1)@^{|-}\restore}
\newcommand{\po}[1][dr]{\save*!/#1+1.5pc/#1:(1,-1)@^{|-}\restore}
\newcommand{\longhookrightarrow}{\ensuremath{\lhook\joinrel\relbar\joinrel\rightarrow}}

%%%%%%%%%%%%%%%%%%%%%%%%%%%%%%%%%%%%%%%%%%%%%%%%%%%%%%%%

\begin{document}

\title{(Short) title of the paper} {Title of the paper}

\author{F.~Author, S.~Author} {First Author${}^1$ and Second Author${}^2$}

\address{${}^1$Address of the first author\\
${}^2$Address of the second author}

\email{fa@gmail.com${}^1$, sa@gmail.com${}^2$}

\received{???} \revised{} \accepted{???}
\volumenumber{???} \volumeyear{???}

\setcounter{page}{1}

\numberwithin{equation}{section}

\maketitle

\begin{abstract}
The text of the abstract.
\end{abstract}

\metadata{18B40}{18C15, 18D05}{monadic category, internal groupoid, bicategory of fractions, axiom of choice}



\section{Introduction}

The text of the introduction. Some citations \cite{baez-crans, benabou, borceux1, bunge-pare, gabriel-zisman}. Math formulas, equations, diagrams
\begin{equation}
\text{Grpd}(\text{Gp}) \longhookrightarrow \text{MON},
\end{equation}
and so on.

\section{Weak equivalences} \label{section weak equivalences}

The text of the second section.

We recall in this section the definition of weak equivalences in the 2-category of internal groupoids.
Let us first fix notations. If $\C$ is a category with pullbacks, an internal groupoid $\A$ in $\C$ is represented by
\begin{equation}
\cd{A_1 \times_{c,d} A_1 \ar[r]^-{m} & A_1 \ar@<2pt>[r]^-{d} \ar@<-2pt>[r]_-{c} & A_0 \ar@<2pt>@/^1pc/[l]^-{e}} \qquad \cd{A_1 \ar[r]^-{i} & A_1}
\end{equation}
where
\begin{equation}
\cd{A_1 \times_{c,d} A_1 \pb \ar[r]^-{\pi_2} \ar[d]_-{\pi_1} & A_1 \ar[d]^-{d} \\ A_1 \ar[r]_-{c} & A_0}
\end{equation}
is a pullback.



The definition of weak equivalences has been introduced in \cite{bunge-pare}.

\begin{definition}
Let $\C$ be a category with pullbacks and $F: \A \rightarrow \B$ be an internal functor between internal groupoids in $\C$ ...
\end{definition}



\begin{theorem} \label{preserving weak equivalences}
Let $U: \D \rightarrow \C$ be a pullback preserving functor between categories with pullbacks and let $F : \A \rightarrow \B$
be an internal functor between internal groupoids in $\D$.
Then,
\begin{itemize}
\item If $U$ reflects finite limits, $F$ is full and faithful if and only if $UF$ is.
\item If $\C$ and $\D$ are regular and if $U$ preserves and reflects regular epimorphisms, then $F$ is essentially surjective if and only if $UF$ is.
\end{itemize}
\end{theorem}

\begin{proof}
The `only if parts' follow from the preserving hypothesis while the `if parts' follow from the reflecting hypothesis.
\end{proof}

We conclude this section by a well-known lemma.

\begin{lemma}
Let $\C$ be a category with pullbacks. An internal functor $F: \A \rightarrow \B$ between internal groupoids in $\C$ is an equivalence if and only if it is full and faithful and
$$\cd{A_0 \times_{F_0,d} B_1 \ar[r]^-{t_2} & B_1 \ar[r]^-{c} & B_0}$$
is a split epimorphism.
\end{lemma}



\section{$\T$-monoidal functors} \label{section T-monoidal functors}

The text of the third section.


\begin{corollary} \label{corollary for monadic categories}
Let $\C$ be a regular category where the axiom of choice holds and $G: \D \rightarrow \C$ be a monadic functor.
Denote by $\T=(T,\eta, \mu)$ the monad induced by the adjunction $F \dashv G$ on $\C$ and $K: \D \rightarrow \C^{\T}$ the comparison functor.
Then, the composite
\begin{equation}
\cd{\text{Grpd}(\D) \ar[r]^-{K} & \text{Grpd}(\C^{\T}) \; \ar@{^{(}->}[r]^-{I} & \T \text{-MON}}
\end{equation}
is the bicategory of fractions of $\text{Grpd}(\D)$ with respect to weak equivalences.
\end{corollary}

\begin{proof}
Since $G$ is monadic, $K: \D \rightarrow \C^{\T}$ is an equivalence and $K: \text{Grpd}(\D) \rightarrow \text{Grpd}(\C^{\T})$ is a
biequivalence of 2-categories.
In addition, by Lemma \ref{preserving weak equivalences}, $W \in \text{Grpd}(\D)$ is a weak equivalence if and only if $K(W) \in \text{Grpd}(\C^{\T})$ is.
Thus, $IK$ satisfies conditions EF1, EF2 and EF3.
\end{proof}



\begin{thebibliography}{12}


\bibitem{baez-crans}
{J. C. Baez and A. S. Crans}, \textit{Higher-dimensional algebra VI: Lie 2-algebras}, {Theory and Applications of Categories}
\textbf{12} (2004) 492--538.

\bibitem{benabou}
{J. B\'enabou}, \textit{Introduction to bicategories}, {Springer LNM} \textbf{40} (1967) 1--77.

\bibitem{borceux1}
{F. Borceux}, \textit{ Handbook of Categorical Algebra 1}, {Cambridge University Press} (1994).

\bibitem{bunge-pare}
{M. Bunge and R. Par\'e}, \textit{Stacks and equivalence of indexed categories}, {Cahiers de Topologie et G\'eom\'etrie Diff\'erentielle
Cat\'egoriques}
\textbf{20} (1979) 373--399.

\bibitem{gabriel-zisman}
{P. Gabriel and M. Zisman}, \textit{Calculus of fractions and homotopy theory}, {Springer} (1967).


\end{thebibliography}

\end{document}
